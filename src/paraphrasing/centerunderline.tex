\setul{0.3ex}{0.3ex} %1te ist der Abstand zum Text 2te ist die Dicke der linie. ex ist eine von der Buchstanbengröße abhängige Maßeinheit.

%Hier definieren wir die unterstreichende Linie.
\newcommand{\numbering}{}			% Ist die Variable, in der immer die Kapitelnummer gespeichert wird
\newcommand{\setNumbering}[1]{		% Hier wird diese gesetzt, da wir sie nicht als Parameter übergeben können.
	\renewcommand{\numbering}{#1}	% Also übergeben an den ulined Befehl. Da wir nacher ja \ulined verwenden.
}
\newcommand{\getColor}{}			% Ist die Variable, in der die Farbe gespeichert wird.
\newcommand{\setColor}[1]{			% Hier wird diese gesetzt, da wir sie nicht als Parameter übergeben können.
	\renewcommand{\getColor}{#1}	% Also übergeben an den ulined Befehl. \ulined nimmt das was dahinter steht.
}
\newcommand{\ulined}[1]{%
    \setLength{#1}%Hier berechnen wir die Länge der gegebenen Überschrift. paraphrasing.tex
    	\begin{tikzpicture}
        	\draw[color=\getColor!50!, fill=\getColor!50!]	%Hier wird die Farbe für den Hintergrund gemischt.
				(-1ex + -0.5 * \length, -0.05 em) -- 	% Das ist der linke Punkt des Polygons. -1ex für gleichen Abs
				(1ex + 0.5 * \length, -0.05 em) --			% Das ist der rechte Punkt der oberen Linie
				(0.55 * \length, -0.75 ex) --			% Das ist der rechte Punkt der unteren Linie. 0.55 für Schräg
				(-0.55 * \length, -0.75 ex) --			% Das ist der linke Punkt der unteren Linie.
				cycle;									% Damit wir wieder zum linken Punkt der oberen Linie springen
        	\node[inner sep=1pt, outer sep=0pt] (Text) {% Der Text der Überschrift wird als Knoten gespeichert.
				#1	% In #1 befindet sich der Text der Überschrift
			};
    	\end{tikzpicture}
}
%Hier stellen wir nur noch ein, dass die oben definierte linie zum untersteichen verwendet werden soll.

\definecolor{myRed}{RGB}{205,85,85}
\titleformat{\section}[hang]
    {\vspace{-3ex} \normalfont \bfseries \Huge \centering}
	{}
	{-8ex}
	{\setColor{myRed} \setNumbering{\thesection} \ulined}
    [\vspace{-2ex}]

\definecolor{myBlue}{RGB}{30,144,255}
\titleformat{\subsection}[hang]
	{\vspace{-2.75ex} \normalfont \bfseries \large}
	{}
	{-6ex}
	{\setColor{myBlue} \setNumbering{\thesubsection} \ulined}
	[\vspace{-1.5ex}]

\definecolor{myGreen}{RGB}{46,139,87}
\titleformat{\subsubsection}[hang]
	{\vspace{-2.5ex} \normalfont \bfseries}
	{}
	{-4ex}
	{\setColor{myGreen} \setNumbering{\thesubsubsection} \ulined}
	[\vspace{-1.5ex}]

\titleformat{\paragraph}[hang]
	{\normalfont \bfseries}
	{}
	{.5em}
	{\enspace \enspace}
	[]
