%\usepackage[T1]{fontenc}
%\usepackage{tgadventor}
%\usepackage{tgheros}
%\usepackage{helvet}
%\usepackage{tgcursor}
%\usepackage{courier}
%\usepackage{xcharter}
%\usepackage{bookman}
%\usepackage{palatino}
%\usepackage{fourier}
%\usepackage{mathptmx}
%\usepackage{lmodern}
%\usepackage{tgbonum}
%\usepackage{titlesec}
\newcommand{\setParaphrasingPath}[1]{%
    \newcommand{\paraphrasingPath}{#1}
    %Da wir später z.B. in PolygonUnderline die Länge eines Arguments brauchen:
    %Und da wir die Länge vermutlich sonst nicht brauchen taucht das sonst nicht auf.
    \newlength{\internerVariablenNameLength}
    \newcommand{\length}{}
}
%Diesen Befehl müssen wir leider auslagern. Die dafür wichtigen Variablen wurden schon in den Zeilen 20 und 21 dekliniert.
\newcommand{\setLength}[1]{
    \settowidth{\internerVariablenNameLength}{#1}
    \renewcommand{\length}{\the\internerVariablenNameLength}
}
\newcommand{\setParaphrasing}[1]{%
    \input{./\paraphrasingPath/#1}
}
