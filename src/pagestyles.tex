\newcommand{\setPagestylePath}[1]{% Wird hier auch zum initialisieren der "Variablen verwendet."
    \newcommand{\pagestylePath}{#1}
    \newcommand{\defaultFooterRuleWidth}{0pt}
    \newcommand{\defaultHeaderRuleWidth}{0pt}
    \newcommand{\sectionName}{}
    \newcommand{\subSectionName}{}
    \newcommand{\currentPagestyle}{}
}
\newcommand{\setPagestyle}[1]{%
    \input{./\pagestylePath/#1}
    \thispagestyle{#1} %Brauchen wir noch für die erste Seite. (27.05.22)
    \AtBeginShipout { %Sollte den folgenden Code auf jeder Seite ausführen und damit den Pagestyle Bug beheben(27.05.22)
        \thispagestyle{#1}
    }
    \renewcommand{\currentPagestyle}{#1}
}
%\setThisPagestyle ist aktuell nicht verwendbar, da AtBeginShipout immer wieder den Pagestyle überschreibt.
% Das ist zwar ein Problem aber eigentlich nur beim Anhang. Dazu haben wir schon eine Lösung gefunden.
% Siehe commands anhang.(27.05.22)
\newcommand{\setThisPagestyle}[1]{%
    \input{./\pagestylePath/#1}
    \thispagestyle{#1}
    \renewcommand{\currentPagestyle}{#1}
}
\newcommand{\setDefaultHeaderRuleWidth}[1]{%
    \renewcommand{\defaultHeaderRuleWidth}{#1}
}
\newcommand{\setDefaultFooterRuleWidth}[1]{%
    \renewcommand{\defaultFooterRuleWidth}{#1}
}
\newcommand{\newSection}[1]{%
    \section{#1}
    \label{\thesection}
    \resetTitledFrameEnvironmentCounters
    \renewcommand{\sectionName}{#1}
    %Ergibt natürlich nur dann Sinn, wenn wir auch die Titled-Framed-Environments verwenden.
}
\newcommand{\newSectionStar}[1]{%
    \section*{#1}
    \resetTitledFrameEnvironmentCounters
    \renewcommand{\sectionName}{#1}
}
\newcommand{\newSubSection}[1]{%
    \subsection{#1}
    \label{\thesubsection}
    \resetTitledFrameEnvironmentCounters
    \renewcommand{\subSectionName}{#1}
    %Ergibt natürlich nur dann Sinn, wenn wir auch die Titled-Framed-Environments verwenden.
}
\newcommand{\newSubSectionStar}[1]{%
    \subsection*{#1}
    \resetTitledFrameEnvironmentCounters
    \renewcommand{\subSectionName}{#1}
}
\newcommand{\newSubSubSection}[1]{%
    \subsubsection{#1}
    \label{\thesubsubsection}
}
