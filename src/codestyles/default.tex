\definecolor{codegray}{rgb}{0.5,0.5,0.5}
\definecolor{intelliJBlau}{RGB}{14,54,170}
\definecolor{intelliJAnnotations}{RGB}{158,143,23}
\definecolor{intelliJStrings}{RGB}{13,130,19}
\definecolor{intelliJBackground}{RGB}{255,255,255}
\definecolor{intelliJVariables}{RGB}{129,6,138}
\definecolor{intelliJMethodNames}{RGB}{0,109,147}
%\definecolor{intelliJComments}{RGB}{}

\lstdefinestyle{default}{
    language=Java,
    backgroundcolor=\color{intelliJBackground},   
    commentstyle=\color{codegray},
    keywordstyle=\color{intelliJBlau},
    numberstyle=\tiny\color{codegray},
    stringstyle=\color{intelliJStrings},
    %basicstyle=\ttfamily\footnotesize,
    %basicstyle=\ttfamily,
    %basicstyle=\footnotesize,
    basicstyle=\small,
    breakatwhitespace=false,         
    breaklines=true,                 
    captionpos=b,                    
    keepspaces=true,                 
    numbers=left,                    
    numbersep=5pt,                  
    showspaces=false,                
    showstringspaces=false,
    showtabs=false,                  
    tabsize=2,
    moredelim = [l][\color{intelliJAnnotations}]{@},
    moredelim = [is][\color{intelliJMethodNames}]{\\(}{\\)},
    moredelim = [is][\color{intelliJVariables}]{\\<}{\\>}
}
%Wie wir sehen fehlen hier noch für Methodennamen und Variablen noch
%die passenden Farben, die können wir noch mit passenden Delimitern
%festlegen. Zum Beispiel mit:
% moredelim = [is][\color{intelliJMethodNames}]{\\(}{\\)}
% moredelim = [is][\color{intelliJVariables}]{\\<}{\\>}
%Das wird beim einlesen von Dokumenten aber echt schwer...

% Wird nicht verwendet, habe nicht herausgefunden was hier schief
% läuft :/
%% #1 ist der Pfad zur Datei "Die Artifact-Id und Group-Id"
%% #2 ist der Name der Datei
%\newcommand{\useSpringBootPath}[2]{%
%    \newcommand{\artifactID}{\getListMember{1}{#1}{/}}
%    \expandafter\renewcommand{\examplepath}{lib/\artifactID/src/main/java/#1/#2.java}
%}

\setEnvironment[default]{Code}{gray-white}

% #1 sind die Keyword-Variablen wie firstline=...
% #2 ist der Name der .java Datei
% #3 ist der Pfad zur Datei
\newcommand{\includeJava}[3][]{%
    \hfill
    \allowbreak
    \vspace{1pt}
    \begin{minipage}[htp]{\linewidth}
        \begin{Code}{#2}
            \vspace{-1em}
            \hspace{0.01\textwidth}
            \begin{minipage}[htp]{0.99\textwidth}
                \lstinputlisting[#1]{lib/#3/#2.java}
            \end{minipage}
            \vspace{-1em}
        \end{Code}
    \vspace{0pt} %Ja diese Zeile macht tatsächlich einen Unterschied
    \end{minipage}
}
