% Wichtig in Zeile 12 den Namen anpassen!

\pgfkeys{%
    /lstlisting/.unknown/.code = { \relax },
    /lstlisting/.cd,
        firstline/.code = {%
            \ifx#1\empty\else
                \def\lstlistingFirstline{#1}
            \fi
        },
        lastline/.code = {%
            \ifx#1\empty\else
                \def\lstlistingLastline{#1}
            \fi
        },
        linerange/.code = {%
            \ifx#1\empty\else
                \def\lstlistingLinerange{#1}
            \fi
        },
        language/.code = {%
            \ifx#1\empty\else
                \def\lstlistingLanguage{#1}
            \fi
        },
        filetype/.code = {%
            \ifx#1\empty\else
                \def\lstlistingFiletype{#1}
            \fi
        }
}


\definecolor{codegray}{rgb}{0.5,0.5,0.5}
\definecolor{intelliJBlau}{RGB}{14,54,170}
\definecolor{intelliJAnnotations}{RGB}{158,143,23}
\definecolor{intelliJStrings}{RGB}{13,130,19}
\definecolor{intelliJBackground}{RGB}{255,255,255}
\definecolor{intelliJVariables}{RGB}{129,6,138}
\definecolor{intelliJMethodNames}{RGB}{0,109,147}
%\definecolor{intelliJComments}{RGB}{}

\lstdefinestyle{springboot}{
    %language=Java,
    backgroundcolor=\color{intelliJBackground},   
    commentstyle=\color{codegray},
    keywordstyle=\color{intelliJBlau},
    numberstyle=\tiny\color{codegray},
    stringstyle=\color{intelliJStrings},
    %basicstyle=\ttfamily\footnotesize,
    %basicstyle=\ttfamily,
    %basicstyle=\footnotesize,
    basicstyle=\small,
    breakatwhitespace=false,         
    breaklines=true,                 
    captionpos=b,                    
    keepspaces=true,                 
    numbers=left,                    
    numbersep=5pt,                  
    showspaces=false,                
    showstringspaces=false,
    showtabs=false,                  
    tabsize=2,
    moredelim = [l][\color{intelliJAnnotations}]{@},
    moredelim = [is][\color{intelliJMethodNames}]{\\(}{\\)},
    moredelim = [is][\color{intelliJVariables}]{\\<}{\\>}
}
%Wie wir sehen fehlen hier noch für Methodennamen und Variablen noch
%die passenden Farben, die können wir noch mit passenden Delimitern
%festlegen. Zum Beispiel mit:
% moredelim = [is][\color{intelliJMethodNames}]{\\(}{\\)}
% moredelim = [is][\color{intelliJVariables}]{\\<}{\\>}
%Das wird beim einlesen von Dokumenten aber echt schwer...

% Wird nicht verwendet, habe nicht herausgefunden was hier schief
% läuft :/
%% #1 ist der Pfad zur Datei "Die Artifact-Id und Group-Id"
%% #2 ist der Name der Datei
%\newcommand{\useSpringBootPath}[2]{%
%    \newcommand{\artifactID}{\getListMember{1}{#1}{/}}
%    \expandafter\renewcommand{\examplepath}{lib/\artifactID/src/main/java/#1/#2.java}
%}

\setEnvironment[default]{Code}{gray-white}

\NewDocumentCommand{\includeSpringQuery}{O{}mm}{%
    \lstset{language=SQL}
    \def\lstlistingFirstline{}
    \def\lstlistingLastline{}
    \def\lstlistingLinerange{}
    \pgfkeys{/lstlisting/.cd, #1}
    \hfill
    \allowbreak
    \vspace{1pt}
    \parbox{\linewidth}{%
        \begin{Code}[#1]{#2}
            \vspace{-1em}
            \hspace{0.01\textwidth}
                \ifx\lstlistingFirstline\empty
                    \ifx\lstlistingLastline\empty
                        \ifx\lstlistingLinerange\empty
                            \lstinputlisting{%
                                lib/#3/src/main/resources/querys/#2.sql
                            }
                        \else
                            \lstinputlisting[%
                                linerange=\lstlistingLinerange
                            ]{%
                                lib/#3/src/main/resources/querys/#2.sql
                            }
                        \fi
                    \else
                        \ifx\lstlistingLinerange\empty
                            \lstinputlisting[lastline=\lstlistingLastline]{}
                        \else
                            \lstinputlisting[%
                                lastline=\lstlistingLastline,
                                linerange=\lstlistingLinerange
                            ]{%
                                lib/#3/src/main/resources/querys/#2.sql
                            }
                        \fi
                    \fi
                \else 
                    \ifx\lstlistingLastline\empty
                        \ifx\lstlistingLinerange\empty
                            \lstinputlisting[%
                                firstline=\lstlistingFirstline
                            ]{%
                                lib/#3/src/main/resources/querys/#2.sql
                            }
                        \else
                            \lstinputlisting[%
                                firstline=\lstlistingFirstline,
                                linerange=\lstlistingLinerange
                            ]{%
                                lib/#3/src/main/resources/querys/#2.sql
                            }
                        \fi
                    \else
                        \ifx\lstlistingLinerange\empty
                            \lstinputlisting[%
                                firstline=\lstlistingFirstline,
                                lastline=\lstlistingLastline
                            ]{%
                                lib/#3/src/main/resources/querys/#2.sql
                            }
                        \else
                            \lstinputlisting[%
                                firstline=\lstlistingFirstline,
                                lastline=\lstlistingLastline,
                                linerange=\lstlistingLinerange
                            ]{%
                                lib/#3/src/main/resources/querys/#2.sql
                            }
                        \fi
                    \fi
                \fi
            \vspace{-1em}
        \end{Code}
    \vspace{0pt} %Ja diese Zeile macht tatsächlich einen Unterschied
    }
    \citationtext{#1}%

}

% #1 sind die Keyword-Variablen wie firstline=..., author=...
% #2 ist der Name der .java Datei
% #3 ist die Artifact-Id wie in IntelliJ
% #4 ist der Modul-Pfad zur Datei falls vorhanden
\NewDocumentCommand{\includeSpringCode}{O{}mmO{}}{%
    \lstset{language=Java}
    \def\lstlistingFirstline{}
    \def\lstlistingLastline{}
    \def\lstlistingLinerange{}
    \pgfkeys{/lstlisting/.cd, #1}
    \hfill
    \allowbreak
    \vspace{1pt}
    \parbox{\linewidth}{%
        \begin{Code}[#1]{#2}
            \vspace{-1em}
            \hspace{0.01\textwidth}
            \ifx\lstlistingLinerange\empty
                \ifx\lstlistingFirstline\empty
                    \ifx\lstlistingLastline\empty
                        \ifx\relax#4\relax
                            \lstinputlisting{
                                lib/#3/src/main/java/#3/#2.java
                            }
                        \else
                            \lstinputlisting{
                                lib/#3/src/main/java/#3/#4/#2.java
                            }
                        \fi
                    \else
                        \ifx\relax#4\relax
                            \lstinputlisting[
                                lastline=\lstlistingLastline
                            ]{
                                lib/#3/src/main/java/#3/#2.java
                            }
                        \else
                            \lstinputlisting[
                                lastline=\lstlistingLastline
                            ]{
                                lib/#3/src/main/java/#3/#4/#2.java
                            }
                        \fi
                    \fi
                \else
                    \ifx\lstlistingLastline\empty
                        \ifx\relax#4\relax
                            \lstinputlisting[
                                firstline=\lstlistingFirstline
                            ]{
                                lib/#3/src/main/java/#3/#2.java
                            }
                        \else
                            \lstinputlisting[
                                firstline=\lstlistingFirstline
                            ]{
                                lib/#3/src/main/java/#3/#4/#2.java
                            }
                        \fi
                    \else
                        \ifx\relax#4\relax
                            \lstinputlisting[
                                firstline=\lstlistingFirstline,
                                lastline=\lstlistingLastline
                            ]{
                                lib/#3/src/main/java/#3/#2.java
                            }
                        \else
                            \lstinputlisting[
                                firstline=\lstlistingFirstline,
                                lastline=\lstlistingLastline
                            ]{
                                lib/#3/src/main/java/#3/#4/#2.java
                            }
                        \fi
                    \fi
                \fi
            \else
                \ifx\lstlistingFirstline\empty
                    \ifx\lstlistingLastline\empty
                        \ifx\relax#4\relax
                            \lstinputlisting[
                                linerange=\lstlistingLinerange
                            ]{
                                lib/#3/src/main/java/#3/#2.java
                            }
                        \else
                            \lstinputlisting[
                                linerange=\lstlistingLinerange
                            ]{
                                lib/#3/src/main/java/#3/#4/#2.java
                            }
                        \fi
                    \else
                        \ifx\relax#4\relax
                            \lstinputlisting[
                                linerange=\lstlistingLinerange,
                                lastline=\lstlistingLastline
                            ]{
                                lib/#3/src/main/java/#3/#2.java
                            }
                        \else
                            \lstinputlisting[
                                linerange=\lstlistingLinerange,
                                lastline=\lstlistingLastline
                            ]{
                                lib/#3/src/main/java/#3/#4/#2.java
                            }
                        \fi
                    \fi
                \else
                    \ifx\lstlistingLastline\empty
                        \ifx\relax#4\relax
                            \lstinputlisting[
                                linerange=\lstlistingLinerange,
                                firstline=\lstlistingFirstline
                            ]{
                                lib/#3/src/main/java/#3/#2.java
                            }
                        \else
                            \lstinputlisting[
                                linerange=\lstlistingLinerange,
                                firstline=\lstlistingFirstline
                            ]{
                                lib/#3/src/main/java/#3/#4/#2.java
                            }
                        \fi
                    \else
                        \ifx\relax#4\relax
                            \lstinputlisting[
                                linerange=\lstlistingLinerange,
                                firstline=\lstlistingFirstline,
                                lastline=\lstlistingLastline
                            ]{
                                lib/#3/src/main/java/#3/#2.java
                            }
                        \else
                            \lstinputlisting[
                                linerange=\lstlistingLinerange,
                                firstline=\lstlistingFirstline,
                                lastline=\lstlistingLastline
                            ]{
                                lib/#3/src/main/java/#3/#4/#2.java
                            }
                        \fi
                    \fi
                \fi
            \fi
            \vspace{-1em}
        \end{Code}
    \vspace{0pt} %Ja diese Zeile macht tatsächlich einen Unterschied
    }
    \citationtext{#1}%
}

% #1 sind die Keyword-Variablen wie firstline=...
% #2 ist der Name der .java Datei
% #3 ist der Pfad zur .java Datei ab lib/
\newcommand{\includeCode}[3][]{%
    \def\lstlistingFiletype{}
    \def\lstlistingLanguage{}
    \def\lstlistingFirstline{}
    \def\lstlistingLastline{}
    \def\lstlistingLinerange{}
    \pgfkeys{/lstlisting/.cd, #1}
    \ifx\lstlistingLanguage\empty
        \def\lstlistingLanguage{java}
    \fi
    \ifx\lstlistingFiletype\empty
        \def\lstlistingFiletype{\lstlistingLanguage}
    \fi
    \lstset{language=\lstlistingLanguage}
    \hfill
    \allowbreak
    \vspace{1pt}
    \parbox{\linewidth}{%
        \begin{Code}[#1]{#2}
            \vspace{-1em}
            \hspace{0.01\textwidth}
                \ifx\lstlistingFirstline\empty
                    \ifx\lstlistingLastline\empty
                        \ifx\lstlistingLinerange\empty
                            \lstinputlisting{%
                                lib/#3/#2.\lstlistingFiletype
                            }
                        \else
                            \lstinputlisting[%
                                linerange=\lstlistingLinerange
                            ]{%
                                lib/#3/#2.\lstlistingFiletype
                            }
                        \fi
                    \else
                        \ifx\lstlistingLinerange\empty
                            \lstinputlisting[%
                                lastline=\lstlistingLastline
                            ]{%
                                lib/#3/#2.\lstlistingFiletype
                            }
                        \else
                            \lstinputlisting[%
                                lastline=\lstlistingLastline,
                                linerange=\lstlistingLinerange
                            ]{%
                                lib/#3/#2.\lstlistingFiletype
                            }
                        \fi
                    \fi
                \else 
                    \ifx\lstlistingLastline\empty
                        \ifx\lstlistingLinerange\empty
                            \lstinputlisting[%
                                firstline=\lstlistingFirstline
                            ]{%
                                lib/#3/#2.\lstlistingFiletype
                            }
                        \else
                            \lstinputlisting[%
                                firstline=\lstlistingFirstline,
                                linerange=\lstlistingLinerange
                            ]{%
                                lib/#3/#2.\lstlistingFiletype
                            }
                        \fi
                    \else
                        \ifx\lstlistingLinerange\empty
                            \lstinputlisting[%
                                firstline=\lstlistingFirstline,
                                lastline=\lstlistingLastline
                            ]{%
                                lib/#3/#2.\lstlistingFiletype
                            }
                        \else
                            \lstinputlisting[%
                                firstline=\lstlistingFirstline,
                                lastline=\lstlistingLastline,
                                linerange=\lstlistingLinerange
                            ]{%
                                lib/#3/#2.\lstlistingFiletype
                            }
                        \fi
                    \fi
                \fi
            \vspace{-1em}
        \end{Code}
    \vspace{0pt} %Ja diese Zeile macht tatsächlich einen Unterschied
    }
    \citationtext{#1}%
}
