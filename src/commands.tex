% Ermöglicht es aus einer Liste den zu einem Index gehörenden Wert
% auszulesen.
% Einen Usecase gibt es zum Beispiel in codestyles Hier wird der Pfad
% getrennt.
% #1 ist der Index
% #2 ist die Liste
% #3 ist der Seperator der Werte der Liste
\newcommand{\getListMember}[3]{\StrBetween[#1,\number\numexpr#1+1]{#3#2#3}{#3}{#3}}

\newcommand{\blankPage}{
    \newpage
    \setThisPagestyle{plain}
    \addtocounter{page}{-1}
    \null
    \newpage
}

% Folgende Befehle machen Probleme:
%\newcommand{\enlargePage}[1]{
%    \newgeometry{top=4.5cm - #1/2, bottom=4.5cm - #1/2}
%}
%\newcommand{\restorePageSize}{
%    \clearpage
%    \restoregeometry
%}
% mark wird wohl in zwischen von einem anderen Packet verwendet
%\renewcommand{\mark}[1]{\glqq #1\grqq{}}
\newcommand{\tikzmark}[1]{\tikz[baseline, remember picture] \coordinate (#1) {};}
%\newcommand{\bold}[1]{\textbf{#1}}

%Um das richtige Datumsformat zu haben
\newcommand{\leadingzero}[1]{\ifnum #1<10 0\the#1\else\the#1\fi}
\newcommand{\datumVonHeute}{\leadingzero{\day}.\leadingzero{\month}.\the\year}
\newcommand{\anhang}{
    \newpage
    \refstepcounter{section}%
    \setcounter{subsection}{0}
    \setcounter{subsubsection}{0}
    \addcontentsline{toc}{section}{\protect\numberline{\thesection}{Anhang}}
    \renewcommand{\subSectionName}{}
    \renewcommand{\sectionName}{Anhang}
    \phantomsection
    \hypertarget{Anhang}
    %\renewcommand{\subSectionName}{Aufgaben zum Einstieg}
%\includepdf[
%    landscape,
%    fitpaper = false,
%    %trim = 35mm 10mm 15mm 15mm, %Hier definieren wir wie viel wir links,  unten, rechts oder oben abschneiden.
%    noautoscale = false,
%    scale=.95,
%    pages={1,2},
%    pagecommand={},
%    addtotoc={
%        1,subsection,1,Aufgaben zum Einstieg,AZE
%    }
%]
%{lib/PDF/HeinrichWinter.pdf}

}
\newcommand{\lit}{
    \newpage
    %\refstepcounter{section}%
    \setcounter{subsection}{0}
    \setcounter{subsubsection}{0}
    %\addcontentsline{toc}{section}{\protect\numberline{\thesection}{Literatur}}
    \renewcommand{\subSectionName}{}
    %\renewcommand{\sectionName}{Literatur}
    \newSection{Literatur}
    \printbibliography[heading=none]
}
\newcommand{\checkedSquare}{\mbox{\ooalign{$\checkmark$\cr\hidewidth$\square$\hidewidth\cr}}}
\newcommand{\R}{\ensuremath{\mathbb{R}}}
\newcommand{\N}{\ensuremath{\mathbb{N}}}
\newcommand{\footQuote}[3]{\glqq #1\grqq{}\footcite[#3]{#2}}
\newcommand{\newFootRef}[3]{\glqq #1\grqq{} \hspace{-1.25ex} \footnotemark \footnotetext{\hspace{1ex} \hyperref[#3]{#2}}}
\newcommand{\toroman}[1]{\romannumeral #1 }
\newcommand{\toRoman}[1]{\MakeUppercase{\romannumeral #1}}
\renewcommand{\newline}{\vspace{1em}\\}
\newcommand{\newQuote}[3]{\glqq #1\grqq{}\footcite[#3]{#2}}
\newcommand{\floor}[1]{\left\lfloor #1 \right\rfloor{}}

%\censorruledepth = -.2ex
%\censorruleheight = .1ex
\newcommand{\luecke}[1]{
    %\xblackout{\LARGE #1}
    \newdimen\mywidth
    \setbox0=\vbox{#1}
    \mywidth=\wd0
    %\settowidth{\width0}{\widthof{#1}}
    %\sbox{\textbox}{#1}
    \phantom{\pbox{\mywidth}{#1}}
}
\newcommand{\textluecke}[1]{
    \underline{\luecke{\LARGE #1}}
}
\newcommand{\tl}[1]{
    \textluecke{#1}
}

%Hier ergänze ich noch eine Varriante, wie ein Zitat zu Beginn eines Kapitels aussehen könnte.
\newenvironment{chapquote}[2][2em]
  {\setlength{\@tempdima}{#1}%
   \def\chapquote@author{#2}%
   \parshape 1 \@tempdima \dimexpr\textwidth-2\@tempdima\relax%
   \itshape}
  {\par\normalfont\hfill--\ \chapquote@author\hspace*{\@tempdima}\par\bigskip}

\DeclareUnicodeCharacter{202F}{FIX ME!!!!}

%Wurden in init verschoben könnten aber auch hier verwendet werden richtiger Anker ist wichtig.

% Nur eine Beta-Version hier muss noch weiter dran gearbeitet werden. Winkel stimmt nicht.
%
%%\usetikzlibrary{intersections}
%%\usetikzlibrary{calc}
%\newcommand{\test}[4]{
%    \path[name path = rim] (#2) circle (1cm);
%    \path[name path = toA] (#2) -- (#1);
%    \path[name path = toB] (#2) -- (#3);
%    \path[name intersections = {of = rim and toA}];
%    \path[name intersections = {of = rim and toB}];
%    \pgfmathanglebetweenpoints{%
%        \pgfpointanchor{intersection-1}{center}
%    }
%    {%
%        \pgfpointanchor{intersection-2}{center}
%    }
%    \draw[dashed, very thick, red] (isectionA) arc (\pgfmathresult:360-\pgfmathresult:#4);
%}
%

%\tiny
%\scriptsize
%\footnotesize
%\small
%\normalsize
%\large
%\Large
%\LARGE
%\huge
%\Huge
%\slshape %\sansserief schrift
