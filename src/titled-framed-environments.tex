\newcommand{\setTitledFramePath}[1]{%
    \newcommand{\titledFramePath}{#1}
    \newcommand{\setFrameLabelStyle}{}
    \newcommand{\setFrameTitle}{} %definiert getFrameTitle
    \newcommand{\getFrameTitle}{} %in frame-title-styles definiert.
    \newcommand{\counterList}{}
    \newcommand{\setEnvironmentStyle}{}
}

\newcommand{\setEnvironment}[3][default]{%
%
    %#1 ist der Stil des Titels (hauptsächlich wirkt sich das auf die
    %Nummerierung aus. Ist aber ein optionaler Parameter und wird,
    %wenn nicht angegeben auf default gesetzt.)
    %
    %#2 ist der Name der Umgebung z.b. Definition oder Satz.
    %#3 ist der Stil der Umgebung, also z.B. light-blue oder
    %gray-white.
%
    \newcounter{#2}
    \listadd{\counterList}{#2}
    \input{./\titledFramePath/frame-title-label-styles/#1}
    \setFrameLabelStyle{#2}
    \newenvironment{#2}[2][#1] 
    %
        %##1 beschreibt den Stil des Titels (hauptsächlich wirkt sich
        %das auf die Nummerierung aus.) Es ist ein optionaler
        %Parameter und solte er nicht angegeben werden, so wird er auf
        %#1 gesetzt. #1 beschreibt genau das selbe, wird nur früher
        %schon gesetzt. (in init.tex)
        %
        %##2 ist der Inhalt des Titels.
        %
        % Für weitere informationen:
        % https://tex.stackexchange.com/questions/42463/what-is-the-meaning-of-double-pound-symbol-number-sign-hash-character-1-in
    %
    {
        %begin-code
        \input{./\titledFramePath/frame-title-styles/##1}
        \setFrameTitle{#2}{##2}
        \input{./\titledFramePath/#3}
        \setEnvironmentStyle{#3}
        \begin{#3}
    }
    {
        %end-code
        \end{#3}
    }
}
\newcommand{\resetTitledFrameEnvironmentCounters}{%
    \renewcommand{\do}[1]{%
        \setcounter{##1}{0}
    }
    \dolistloop{\counterList}
}
