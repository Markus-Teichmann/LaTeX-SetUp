% pgfkeys ermöglicht es keyword Argumente an einen Befehl zu
% übergeben. z.B. \citationText{author=Cosmina, note=vgl. S.123} um
% auf diese Keywords jetzt zuzugreifen wird im folgendem ein passender
% Befehl definiert. z.B. \citationAuthor der dann den Wert (hier
% Cosmina) erhält. Dies funktioniert durch den Aufruf von
% \pgfkeys{/citation/.cd, #1} in \citationtext. Weiter kann dann in
% \citationtext über die definierten Befehle \citationAuthor und
% \citationNote auf die übergebenen Werte also "Cosmina" und "vgl.
% S.123" zugreifen.
\pgfkeys{%
    /titledFramedEnvironments/.unknown/.code = { \relax },
    /titledFramedEnvironments/.cd,
        titlestyle/.code = {%
            \ifx#1\empty\else
                \def\titledFramedEnvironmentsTitlestyle{#1}
            \fi
        }
}

\newcommand{\setTitledFramePath}[1]{%
    \newcommand{\titledFramePath}{#1}
    \newcommand{\setFrameLabelStyle}{}
    \newcommand{\setFrameTitle}{} %definiert getFrameTitle
    \newcommand{\getFrameTitle}{} %in frame-title-styles definiert.
    \newcommand{\counterList}{}
    \newcommand{\setEnvironmentStyle}{}
}

\newcommand{\setEnvironment}[3][]{%
%
    % #1 ist ein Optionaler Parameter. Er enthält eine Liste von mit
    % einem Komma getrennten Key-Value Paaren, die mittels pgfkeys
    % passend benannte macros definieren, die im Verlaufe des Codes
    % dann an passenden Stellen ihre Anwendung finden.
    % Hier besonders relevant ist das Key-Value paar
    % titlestyle=default. Es beschreibt den Stil des Titels
    % (hauptsächlich wirkt sich das auf die
    % Nummerierung aus. Der Default wert des Paars ist default und
    % abgerufen kann der Wert mit dem Befehl \titledFramedEnvironmentsTitlestyle
    %
    %#2 ist der Name der Umgebung z.b. Definition oder Satz.
    %#3 ist der Stil der Umgebung, also z.B. light-blue oder
    %gray-white.
%
%
    % In folgendem wollen wir die optionale Variable weiter ausnutzen,
    % indem wir durch pgfkeys die Möglichkeit erhalten weitere Keyword
    % arguments anzugeben, wie zum Beispiel eine Referenz zu einem
    % Author oder ähnlichem. Hier interessiert uns aber nur der
    % Titelstil also wie das aussieht und nicht, der Inhalt, darum
    % wird sich in anderen Dateien wie z.B. Citationstyles gekümmert.
    \pgfkeys{/titledFramedEnvironments/.cd, #1}
    \ifdefined\titledFramedEnvironmentsTitlestyle\else
        \def\titledFramedEnvironmentsTitlestyle{default}
    \fi
    % Sollte \titledFramedEnvironmentsTitlestyle vor diesem Code nicht
    % definiert gewesen sein, weil er zum Beispiel in #1 nicht genannt
    % wurde, so ist er spätestens nach diesem Code mit dem Defaultwert
    % default definiert.
%
    \newcounter{#2}
    \listadd{\counterList}{#2}
    \input{./\titledFramePath/frame-title-label-styles/\titledFramedEnvironmentsTitlestyle}
    \setFrameLabelStyle{#2}
    \NewDocumentEnvironment{#2}{O{}m}
    %\newenvironment{#2}[2][] 
    %
        % ##1 enthält eine Liste von Key-Value Paaren wie zum Beispiel
        % titlestyle = default die sich mithilfe von pgfkeys auf
        % verschiedene Dinge auswirken. Der Wert von titlestyle definiert
        % den Befehl \titledFramedEnvironmentsTitlestyle neu. Dieser  
        % beschreibt den Stil des Titels (hauptsächlich wirkt sich das auf
        % die Nummerierung aus.) Der Befehl ist oben schon definiert,
        % sollte hier also nichts angegeben sein, so wird die
        % vorangegangene Definition verwendet. Die vorangegangene
        % Definition wird in der init Datei gesetzt.
        %
        %##2 ist der Inhalt des Titels.
        %
        % Für weitere informationen:
        % https://tex.stackexchange.com/questions/42463/what-is-the-meaning-of-double-pound-symbol-number-sign-hash-character-1-in
    %
    {
        %begin-code
        \pgfkeys{/titledFramedEnvironments/.cd, ##1}
        \input{./\titledFramePath/frame-title-styles/\titledFramedEnvironmentsTitlestyle}
        \setFrameTitle[##1]{#2}{##2}
        \input{./\titledFramePath/#3}
        \setEnvironmentStyle{#3}
        \begin{#3}
    }
    {
        %end-code
        \end{#3}
        \citationtext{##1}
    }
}
\newcommand{\resetTitledFrameEnvironmentCounters}{%
    \renewcommand{\do}[1]{%
        \setcounter{##1}{0}
    }
    \dolistloop{\counterList}
}
