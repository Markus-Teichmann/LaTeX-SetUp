\newcommand{\tikzmark}[1]{\tikz[baseline, remember picture] \coordinate (#1) {};}

\AfterEndPreamble{
    \usetikzlibrary{calc}
    \usetikzlibrary{intersections}
    \usetikzlibrary{angles}
    \usetikzlibrary{quotes}
}

% Da es in tikz noch keinen "guten" Befehl für einen Kreis gibt, war
% das ein erster Versuch selbst einen zu schreiben.
% Nur eine Beta-Version hier muss noch weiter dran gearbeitet werden. Winkel stimmt nicht.
% 
%%\usetikzlibrary{intersections}
%%\usetikzlibrary{calc}
%\newcommand{\test}[4]{
%    \path[name path = rim] (#2) circle (1cm);
%    \path[name path = toA] (#2) -- (#1);
%    \path[name path = toB] (#2) -- (#3);
%    \path[name intersections = {of = rim and toA}];
%    \path[name intersections = {of = rim and toB}];
%    \pgfmathanglebetweenpoints{%
%        \pgfpointanchor{intersection-1}{center}
%    }
%    {%
%        \pgfpointanchor{intersection-2}{center}
%    }
%    \draw[dashed, very thick, red] (isectionA) arc (\pgfmathresult:360-\pgfmathresult:#4);
%}
%
