% Um auch zwischen Section's und SubSection's zu referenzieren habe
% ich in pagestyles.tex ein \label mit der
% Section/Subectionnumber angegeben. Hier definiere ich nur den
% \refsec Befehl.
\newcommand{\refsec}[2]{%
    \hyperref[#2]{#1}
}

% Der Befehl \refenv verlinkt auf die angegebene Umgebung z.B. eine
% Definition: 
% \refenv[Skalarprodukt]{Definition:1.2}
% Dabei ist es wichtig, dem gewählten frame Title Style
% nach zu referenzieren!
\newcommand{\refenv}[2][]{%
    \ifx\relax#1\relax%
        \hyperlink{#2}{#2}
    \else%
        \hyperlink{#2}{#1}
    \fi%
}
\renewcommand{\quote}[3]{%
    \glqq #1\grqq{}
    \hspace{-1ex}
    \footnotemark
    \hspace{-1ex}
    \footnotetext{%
        \citeauthor{#2}: #3 \cite{#2} \citeurl{#2}
    }
}
\newcommand{\source}[2]{%
    \footnotemark
    \hspace{-1ex}
    \footnotetext{%
        \citeauthor{#1}: #2 \cite{#1} \citeurl{#1}
    }
}


\newcommand{\footQuote}[3]{\glqq #1\grqq{}\footcite[#3]{#2}}
\newcommand{\newFootRef}[3]{\glqq #1\grqq{} \hspace{-1.25ex} \footnotemark \footnotetext{\hspace{1ex} \hyperref[#3]{#2}}}
\newcommand{\newQuote}[3]{\glqq #1\grqq{}\footcite[#3]{#2}}
