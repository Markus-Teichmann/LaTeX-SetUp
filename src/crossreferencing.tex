\newcounter{tabs}
\setcounter{tabs}{0}
\newcounter{rows}
\setcounter{rows}{0}
\AtBeginEnvironment{xltabular}{
    \stepcounter{tabs}
    \setcounter{rows}{0}
}
% Besser nicht verwenden, Konflikt mit hypertarget und hyperlink
%\newcommand{\theTable}{
%    \arabic{tabs}
%}
%\newcommand{\theRow}{
%    \arabic{rows}
%}
\newcolumntype{\label}[1]{#1<{%
    \raisebox{1.5em}{\hypertarget{\arabic{tabs}:\arabic{rows}}{}}
    \phantomsection
    \stepcounter{rows}
}}

\newcolumntype{\blabel}[1]{%
    >{%
        \bfseries
    }#1<{%
        \raisebox{1.5em}{\hypertarget{\arabic{tabs}:\arabic{rows}}{}}
        \phantomsection
        \stepcounter{rows}
    }
}

\newcommand{\reftab}[2]{%
    \hyperlink{#2}{#1}
}
\newcommand{\refctab}[2]{%
    \hyperlink{\arabic{tabs}:#2}{#1}
}

% Folgendes ist nur als Hilfestellung gedacht.
\newcolumntype{\numbered}[1]{>{%
    \arabic{rows} 
}#1}

% Ein zentrierter aber bestimmt langer Typ.
\newcolumntype{C}[1]{>{\centering}p{#1}}

% Um auch zwischen Section's und SubSection's zu referenzieren habe
% ich in pagestyles.tex ein \label mit der
% Section/Subectionnumber angegeben. Hier definiere ich nur den
% \refsec Befehl.
\newcommand{\refsec}[2]{%
    \hyperref[#2]{#1}
}

% Der Befehl \refenv verlinkt auf die angegebene Umgebung z.B. eine
% Definition: 
% \refenv{Definition}{1.2}
% Dabei ist es wichtig, dem gewählten frame Title Style
% nach zu referenzieren!
\newcommand{\refenv}[2]{%
    \hyperlink{#1:#2}{#1 #2}
}
