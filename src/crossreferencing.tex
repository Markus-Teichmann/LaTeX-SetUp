% Um auch zwischen Section's und SubSection's zu referenzieren habe
% ich in pagestyles.tex ein \label mit der
% Section/Subectionnumber angegeben. Hier definiere ich nur den
% \refsec Befehl.
\newcommand{\refsec}[2]{%
    \hyperref[#2]{#1}
}

% Der Befehl \refenv verlinkt auf die angegebene Umgebung z.B. eine
% Definition: 
% \refenv[Skalarprodukt]{Definition:1.2}
% Dabei ist es wichtig, dem gewählten frame Title Style
% nach zu referenzieren!
\newcommand{\refenv}[2][]{%
    \ifx\relax#1\relax%
        \hyperlink{#2}{#2}
    \else%
        \hyperlink{#2}{#1}
    \fi%
}

%\newcommand{\footQuote}[3]{\glqq #1\grqq{}\footcite[#3]{#2}}
%\newcommand{\newFootRef}[3]{\glqq #1\grqq{} \hspace{-1.25ex} \footnotemark \footnotetext{\hspace{1ex} \hyperref[#3]{#2}}}
%\newcommand{\newQuote}[3]{\glqq #1\grqq{}\footcite[#3]{#2}}

%Beispiel:
%    \reftab{Text-Link}{1:2}
\newcommand{\reftab}[2]{%
    \hyperlink{l:#2}{#1}
}

%#1 wird einfach normal gedruckt
%#2 ist die Referenz
%Mit \arabic{tabs} wird auf den aktuellen stand des counters tabs
%zugegriffen. \refctab greift also immer auf die letzte Tabelle zu.
%#2 liefert dann nur die Zeilennummer beginnend bei 0
\newcommand{\refctab}[2]{%
    \hyperlink{l:\thesection.\arabic{tbl}:#2}{#1}
}
