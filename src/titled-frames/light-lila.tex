\renewcommand{\setEnvironmentStyle}[1]{%
    %Title-Section-Color-Settings:
    %\definecolor{frametitlerulecolor}{RGB}{1,0.5,0}
    \definecolor{frametitlebackgroundcolor}{RGB}{150,76,128}
    %Body-Section-Color-Settings
    \definecolor{backgroundcolor}{RGB}{171,102,153}
    \definecolor{linecolor}{RGB}{150,76,128}
    %\definecolor{innerlinecolor}{RGB}{1,0.5,0}
    %\definecolor{middellinecolor}{RGB}{1,0.5,0}
    %\definecolor{outerlinecolor}{RGB}{1,0.5,0}
    \newmdenv[%
    %General-Settings:
        frametitle={\getFrameTitle},
        %roundcorner=10pt,
        subtitlebelowline=true,
        %subtitleaboveline=true,
        %subtitlebackgroundcolor=black!20!white,
    %
    %Title-Section-Settings:
        frametitlerule=true,
        %frametitlerulecolor=frametitlerulecolor,
        frametitlerulewidth=2pt,
        frametitlebackgroundcolor=frametitlebackgroundcolor,
    %
    %Body-Section-Settings:
        backgroundcolor=backgroundcolor,
        linecolor=linecolor,
        linewidth=2pt,
        %innerlinecolor=innerlinecolor,
        %middellinecolor=middellinecolor,
        %outerlinecolor=outerlinecolor,
        %innerlinewidth=,
        %middellinewidth=,
        %outerlinewidth=,
    ]{#1} %#1 ist der aus titled-framed-environment übergebene Style der Umgebung. z.B. light-blue
}
