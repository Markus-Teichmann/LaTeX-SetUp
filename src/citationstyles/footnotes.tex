% pgfkeys ermöglicht es keyword Argumente an einen Befehl zu
% übergeben. z.B. \citationText{author=Cosmina, note=vgl. S.123} um
% auf diese Keywords jetzt zuzugreifen wird im folgendem ein passender
% Befehl definiert. z.B. \citationAuthor der dann den Wert (hier
% Cosmina) erhält. Dies funktioniert durch den Aufruf von
% \pgfkeys{/citation/.cd, #1} in \citationtext. Weiter kann dann in
% \citationtext über die definierten Befehle \citationAuthor und
% \citationNote auf die übergebenen Werte also "Cosmina" und "vgl.
% S.123" zugreifen.
\pgfkeys{%
    /citation/.unknown/.code = { \relax },
    /citation/.cd,
        author/.code = {%
            \ifx#1\empty\else
                \def\citationAuthor{#1}
            \fi
        },
        note/.code = {%
            \ifx#1\empty\else
                \def\citationNote{#1}
            \fi
        }
}

% Die citationmark wird überall dort plaziert, wo auch das zu
% zitierende bzw. das nicht eigene steht.
%
% #1 ist optional kann aber Informationen in Form einer Liste zur
% Quelle enthalten. Zum Beispiel eine Referenz zu einem der .bib -
% files in lib/bibfiles
\newcommand{\citationmark}{%
    \footnotemark
}

\newcommand{\checkedCitationmark}[1]{
    \def\citationAuthor{}
    \def\citationNote{}
    \pgfkeys{/citation/.cd, #1}
    \ifx\citationAuthor\empty\else
        \ifx\citationNote\empty\else
            \ifnum\value{footnote} < 9
                \hspace{-3.5ex}
            \else
                \hspace{-4ex}
            \fi
            \footnotemark
        \fi
    \fi
}

% Der citationtext gibt die Möglichkeit von der citationsmarke weg
% zu springen und noch mehr informationen zu liefern. Zum Beispiel
% Seitenzahlen wären hier gut denkbar.
%
% #1 %ist optional kann aber Informationen in Form einer Liste zur
% Quelle enthalten. Zum Beispiel eine Referenz zu einem der .bib -
% files in lib/bibfiles
\newcommand{\citationtext}[1]{
    \def\citationAuthor{}
    \def\citationNote{}
    \pgfkeys{/citation/.cd, #1}
    \ifx\citationAuthor\empty\else
        \ifx\citationNote\empty\else
            \footnotetext{%
                \citeauthor{\citationAuthor}: %
                \citationNote{} \cite{\citationAuthor} %
                \citeurl{\citationAuthor}%
            }
        \fi
    \fi
}


\renewcommand{\quote}[2][]{%
    \glqq #2\grqq{}
    \hspace{-0.5ex}
    \citationmark
    \hspace{-4.5ex}
    \citationtext{#1}
    %\footnotemark
    %\footnotetext{%
    %    \citeauthor{#2}: #3 \cite{#2} \citeurl{#2}
    %}
}

\newcommand{\source}[1]{%
    \hspace{-0.5ex}
    \citationmark
    \hspace{-4.5ex}
    \citationtext{#1}
    %\footnotemark
    %\footnotetext{%
    %    \citeauthor{#1}: #2 \cite{#1} \citeurl{#1}
    %}
}
