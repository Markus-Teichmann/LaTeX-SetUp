\documentclass[a4paper, 14pt]{article}
\usepackage[left=4cm, right=4cm, top=4cm, bottom=4cm]{geometry}
\usepackage[utf8]{inputenc}
\usepackage[ngerman]{babel}
\usepackage[T1]{fontenc}
%Fonts:
    %\usepackage[default]{comfortaa}
    \usepackage[sfdefault,scaled=.85]{FiraSans}
    %\usepackage{lmodern}
% Tabellenumgebung
    %tabularray baut auf longtable auf
    \usepackage{longtable}  
    \usepackage{tabularray}
% Rahmenumgebung
    %Hier aber auch an anderen Stellen relevant, wegen definecolor
    \usepackage[x11names]{xcolor}
    \usepackage[framemethod=tikz, xcolor]{mdframed}
% Sourcecode - Umgebung
    \usepackage{listings} 
% Crossreferencing
    \usepackage[hyperfootnotes=false]{hyperref}
% Bilder bzw. Floats
    %Ermöglicht es dem Text um die Figures herum zu stehen:
    \usepackage{picinpar}
    \usepackage{wrapfig2}
    %Gibt uns eine FloatBarrier um floates an Stellen festzunageln
    \usepackage[section]{placeins}
    %Bilder selber zeichnen:
    \usepackage{tikz}
    %Um Bilder hinzuzufügen (z.B. svg Dateien ;)
    \usepackage{import}
% Kontrollfluss verändern:
    % Gibt uns Ankerpunkte um Code entsprechend auszuführen.
    \usepackage{atbegshi} 
    % Using IfEndWith in frame-title-styles to determine if the
    % subsection is zero
    \usepackage{xstring}
    % Ermöglicht uns Keyword-Arguments als Parameter zu übergeben.
    \usepackage{pgfkeys}
% Für Paraphrasing - Style:
    % Ermöglicht uns die Farbe und dicke der (unterstreichungsline) zu verändern.
    % Verwendet in paraphrasing/underline.tex
    \usepackage{soul}
    % Grundlage: Ermöglicht uns die Stile für Sections, Subsections...
    \usepackage{titlesec}
% Mathe-Symbole:
    \usepackage{amssymb}
    \usepackage{amsmath}
    \usepackage{mathrsfs}
% Pagestyling:
    \usepackage{fancyhdr}
    % Um in den Header den Platzverbrauch zu regulieren.
    \usepackage{pbox}
% Text gleichmäßig in Spalten aufteilen
    \usepackage[colaction]{multicol}

% Pakete die das CleanUp nicht überstanden haben:
%\usepackage{multirow} % Vermutlich unnötig, da tbl
%\usepackage[modulo]{lineno}
%\usepackage{graphicx} % Da wir schon xcolor verwenden
%\usepackage{color} % Da wir schon xcolor verwenden
%\usepackage{calc} % Bietet Optionen für Counter ist hier aber
%scheinbar obsolet.
%\usepackage{changepage}
%\usepackage{transparent}
%\usepackage{pdfpages}
%\usepackage{newtxsf}
%\usepackage{caption}
%\usepackage{microtype}
%\usepackage{pdflscape}
%\usepackage{setspace}
%\usepackage{xifthen}
%\usepackage{etoolbox}
