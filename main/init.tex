\documentclass[a4paper, 14pt]{article}
\usepackage[left=4cm, right=4cm, top=4cm, bottom=4cm]{geometry}
%Achtung die Werte für top und bottom werden in commands auch verwendet.
\usepackage[utf8]{inputenc}
\usepackage[ngerman]{babel}
\usepackage[T1]{fontenc}
%Fonts:
    %\usepackage[default]{comfortaa}
    \usepackage[sfdefault,scaled=.85]{FiraSans}
\usepackage{newtxsf}
\usepackage{graphicx}
%\usepackage{hyperref}
\usepackage{longtable}  %Für die Longtable-Umgebung tabularray baut darauf auf
\usepackage{caption}
\usepackage[section]{placeins} %Für FloatBarrier um floates an stellen festzunageln
\usepackage{atbegshi} % Behbt den Pagestyle bug
\usepackage[x11names]{xcolor}
%\usepackage{soulutf8} %Ermöglicht uns die Farbe und dicke der (unterstreichungsline) zu verändern. Verwendet in paraphrasing/underline.tex
\usepackage{soul}
\usepackage{amssymb}
\usepackage{amsmath}
\usepackage{mathrsfs}
%\usepackage{pythontex} %Wird bisher nur in titled-framed-environments verwendet.
\usepackage{etoolbox}
%\usepackage{xltabular} %Für variable breite longtable spalten.
\usepackage{microtype}
\usepackage{pdflscape}
%\usepackage{xifthen}
\usepackage{setspace}
\usepackage[modulo]{lineno}
\usepackage[colaction]{multicol}
\usepackage{fancyhdr}
%\usepackage{lmodern}
\usepackage{titlesec}
\usepackage{import}
\usepackage{xifthen}
\usepackage{pdfpages}
\usepackage{transparent}
\usepackage{tikz}
\usepackage[framemethod=tikz, xcolor]{mdframed}
\usepackage{color}
\usepackage{xstring}
\usepackage[hyperfootnotes=false]{hyperref}
\usepackage{multirow}
%\usepackage{plantuml}
\usepackage{pbox} % Wird in den Headdern verwendet.
%\usepackage{censor}
\usepackage{changepage}
\usepackage{calc}
\usepackage{tabularray} % Für die Tabellenumgebung (tblr)
\usepackage{listings} % Für die Code umgebung (lstlisting)

\usepackage[backend=biber, hyperref, style=numeric, citestyle=numeric]{biblatex}
\usepackage{csquotes}

%Für die Abtrennung von Fußnoten zum restlichen Text.
\renewcommand{\footnoterule}{\noindent\smash{\rule[3pt]{\textwidth}{0.4pt}}}

%Für die neue Zeile vor der URL
%Falls in der Lteratur url's angegeben werden.
%%!Achtung, das macht nur bei wirklich langen Url's sinn.
%\appto{\biburlsetup}{\renewcommand*{\UrlFont}{\newline\normalfont\itshape}}

%Einbinden des Literaturverzeichnisses:
\newcommand{\setLibraryPath}[1]{%
    \newcommand{\libraryPath}{#1}
}

\newcommand{\addlibresource}{%
    \addbibresource[glob=true]{./\libraryPath/bibfiles/*.bib}
}

%Wichtig!
%Nach dem ergänzen einer Quelle immer biber main ausführen.

% Folgende Befehle machen Probleme:
%\newcommand{\enlargePage}[1]{
%    \newgeometry{top=4.5cm - #1/2, bottom=4.5cm - #1/2}
%}
%\newcommand{\restorePageSize}{
%    \clearpage
%    \restoregeometry
%}
% mark wird wohl in zwischen von einem anderen Packet verwendet
%\renewcommand{\mark}[1]{\glqq #1\grqq{}}

\renewcommand{\quote}[3]{%
    \glqq #1\grqq{}
    \hspace{-1ex}
    \footnotemark
    \footnotetext{%
        \citeauthor{#2}: #3 \cite{#2} \citeurl{#2}
    }
}
\newcommand{\source}[2]{%
    \footnotemark
    \footnotetext{%
        \citeauthor{#1}: #2 \cite{#1} \citeurl{#1}
    }
}
%\newcommand{\bold}[1]{\textbf{#1}}

%Um das richtige Datumsformat zu haben
\newcommand{\leadingzero}[1]{\ifnum #1<10 0\the#1\else\the#1\fi}
\newcommand{\datumVonHeute}{\leadingzero{\day}.\leadingzero{\month}.\the\year}
\newcommand{\anhang}{
    \newpage
    \refstepcounter{section}%
    \setcounter{subsection}{0}
    \setcounter{subsubsection}{0}
    \addcontentsline{toc}{section}{\protect\numberline{\thesection}{Anhang}}
    \renewcommand{\subSectionName}{}
    \renewcommand{\sectionName}{Anhang}
    \phantomsection
    \hypertarget{Anhang}
    %\renewcommand{\subSectionName}{Aufgaben zum Einstieg}
%\includepdf[
%    landscape,
%    fitpaper = false,
%    %trim = 35mm 10mm 15mm 15mm, %Hier definieren wir wie viel wir links,  unten, rechts oder oben abschneiden.
%    noautoscale = false,
%    scale=.95,
%    pages={1,2},
%    pagecommand={},
%    addtotoc={
%        1,subsection,1,Aufgaben zum Einstieg,AZE
%    }
%]
%{lib/PDF/HeinrichWinter.pdf}

}
\newcommand{\lit}{
    \newpage
    \refstepcounter{section}%
    \setcounter{subsection}{0}
    \setcounter{subsubsection}{0}
    \addcontentsline{toc}{section}{\protect\numberline{\thesection}{Literatur}}
    \renewcommand{\subSectionName}{}
    \renewcommand{\sectionName}{Literatur}
    \printbibliography
}
\newcommand{\checkedSquare}{\mbox{\ooalign{$\checkmark$\cr\hidewidth$\square$\hidewidth\cr}}}
\newcommand{\R}{\ensuremath{\mathbb{R}}}
\newcommand{\N}{\ensuremath{\mathbb{N}}}
\newcommand{\footQuote}[3]{\glqq #1\grqq{}\footcite[#3]{#2}}
\newcommand{\newFootRef}[3]{\glqq #1\grqq{} \hspace{-1.25ex} \footnotemark \footnotetext{\hspace{1ex} \hyperref[#3]{#2}}}
\newcommand{\toroman}[1]{\romannumeral #1 }
\newcommand{\toRoman}[1]{\MakeUppercase{\romannumeral #1}}
\renewcommand{\newline}{\vspace{1em}\\}
\newcommand{\newQuote}[3]{\glqq #1\grqq{}\footcite[#3]{#2}}
\newcommand{\floor}[1]{\left\lfloor #1 \right\rfloor{}}

%\censorruledepth = -.2ex
%\censorruleheight = .1ex
\newcommand{\luecke}[1]{
    %\xblackout{\LARGE #1}
    \newdimen\mywidth
    \setbox0=\vbox{#1}
    \mywidth=\wd0
    %\settowidth{\width0}{\widthof{#1}}
    %\sbox{\textbox}{#1}
    \phantom{\pbox{\mywidth}{#1}}
}
\newcommand{\textluecke}[1]{
    \underline{\luecke{\LARGE #1}}
}
\newcommand{\tl}[1]{
    \textluecke{#1}
}

%Hier ergänze ich noch eine Varriante, wie ein Zitat zu Beginn eines Kapitels aussehen könnte.
\newenvironment{chapquote}[2][2em]
  {\setlength{\@tempdima}{#1}%
   \def\chapquote@author{#2}%
   \parshape 1 \@tempdima \dimexpr\textwidth-2\@tempdima\relax%
   \itshape}
  {\par\normalfont\hfill--\ \chapquote@author\hspace*{\@tempdima}\par\bigskip}

\DeclareUnicodeCharacter{202F}{FIX ME!!!!}

%Wurden in init verschoben könnten aber auch hier verwendet werden richtiger Anker ist wichtig.

% Nur eine Beta-Version hier muss noch weiter dran gearbeitet werden. Winkel stimmt nicht.
%
%%\usetikzlibrary{intersections}
%%\usetikzlibrary{calc}
%\newcommand{\test}[4]{
%    \path[name path = rim] (#2) circle (1cm);
%    \path[name path = toA] (#2) -- (#1);
%    \path[name path = toB] (#2) -- (#3);
%    \path[name intersections = {of = rim and toA}];
%    \path[name intersections = {of = rim and toB}];
%    \pgfmathanglebetweenpoints{%
%        \pgfpointanchor{intersection-1}{center}
%    }
%    {%
%        \pgfpointanchor{intersection-2}{center}
%    }
%    \draw[dashed, very thick, red] (isectionA) arc (\pgfmathresult:360-\pgfmathresult:#4);
%}
%

%\tiny
%\scriptsize
%\footnotesize
%\small
%\normalsize
%\large
%\Large
%\LARGE
%\huge
%\Huge
%\slshape %\sansserief schrift

%\usepackage{import}
%\usepackage{xifthen}
%\usepackage{pdfpages}
%\usepackage{transparent}

\newcommand{\setImagePath}[1]{%
    \graphicspath{ {./#1/} }
    \newcommand{\imagePath}{#1} % Hier wird newcommand zum deklarieren einer Variable "missbraucht"
}

\newcommand{\includesvg}[2]{%
    \def\svgwidth{#1}
    \import{./\imagePath/svg/}{#2.pdf_tex}
}
%ToDo neuer Befehl für includegraphics
\newcommand{\includeScaledsvg}[2]{%
    \def\svgscale{#1}
    \import{./\imagePath/svg/}{#2.pdf_tex}
}

\newcommand{\setPagestylePath}[1]{% Wird hier auch zum initialisieren der "Variablen verwendet."
    \newcommand{\pagestylePath}{#1}
    \newcommand{\defaultFooterRuleWidth}{0pt}
    \newcommand{\defaultHeaderRuleWidth}{0pt}
    \newcommand{\sectionName}{}
    \newcommand{\subSectionName}{}
    \newcommand{\currentPagestyle}{}
}
\newcommand{\setPagestyle}[1]{%
    \input{./\pagestylePath/#1}
    \thispagestyle{#1} %Brauchen wir noch für die erste Seite. (27.05.22)
    \AtBeginShipout { %Sollte den folgenden Code auf jeder Seite ausführen und damit den Pagestyle Bug beheben(27.05.22)
        \thispagestyle{#1}
    }
    \renewcommand{\currentPagestyle}{#1}
}
%\setThisPagestyle ist aktuell nicht verwendbar, da AtBeginShipout immer wieder den Pagestyle überschreibt.
% Das ist zwar ein Problem aber eigentlich nur beim Anhang. Dazu haben wir schon eine Lösung gefunden.
% Siehe commands anhang.(27.05.22)
\newcommand{\setThisPagestyle}[1]{%
    \input{./\pagestylePath/#1}
    \thispagestyle{#1}
    \renewcommand{\currentPagestyle}{#1}
}
\newcommand{\setDefaultHeaderRuleWidth}[1]{%
    \renewcommand{\defaultHeaderRuleWidth}{#1}
}
\newcommand{\setDefaultFooterRuleWidth}[1]{%
    \renewcommand{\defaultFooterRuleWidth}{#1}
}
\newcommand{\newSection}[1]{%
    \section{#1}
    \label{\thesection}
    \resetTitledFrameEnvironmentCounters
    \renewcommand{\sectionName}{#1}
    %Ergibt natürlich nur dann Sinn, wenn wir auch die Titled-Framed-Environments verwenden.
}
\newcommand{\newSectionStar}[1]{%
    \section*{#1}
    \resetTitledFrameEnvironmentCounters
    \renewcommand{\sectionName}{#1}
}
\newcommand{\newSubSection}[1]{%
    \subsection{#1}
    \label{\thesubsection}
    \resetTitledFrameEnvironmentCounters
    \renewcommand{\subSectionName}{#1}
    %Ergibt natürlich nur dann Sinn, wenn wir auch die Titled-Framed-Environments verwenden.
}
\newcommand{\newSubSectionStar}[1]{%
    \subsection*{#1}
    \resetTitledFrameEnvironmentCounters
    \renewcommand{\subSectionName}{#1}
}
\newcommand{\newSubSubSection}[1]{%
    \subsubsection{#1}
    \label{\thesubsubsection}
}

%\usepackage[T1]{fontenc}
%\usepackage{tgadventor}
%\usepackage{tgheros}
%\usepackage{helvet}
%\usepackage{tgcursor}
%\usepackage{courier}
%\usepackage{xcharter}
%\usepackage{bookman}
%\usepackage{palatino}
%\usepackage{fourier}
%\usepackage{mathptmx}
%\usepackage{lmodern}
%\usepackage{tgbonum}
%\usepackage{titlesec}
\newcommand{\setParaphrasingPath}[1]{%
    \newcommand{\paraphrasingPath}{#1}
    %Da wir später z.B. in PolygonUnderline die Länge eines Arguments brauchen:
    %Und da wir die Länge vermutlich sonst nicht brauchen taucht das sonst nicht auf.
    \newlength{\internerVariablenNameLength}
    \newcommand{\length}{}
}
%Diesen Befehl müssen wir leider auslagern. Die dafür wichtigen Variablen wurden schon in den Zeilen 20 und 21 dekliniert.
\newcommand{\setLength}[1]{
    \settowidth{\internerVariablenNameLength}{#1}
    \renewcommand{\length}{\the\internerVariablenNameLength}
}
\newcommand{\setParaphrasing}[1]{%
    \input{./\paraphrasingPath/#1}
}

% pgfkeys ermöglicht es keyword Argumente an einen Befehl zu
% übergeben. z.B. \citationText{author=Cosmina, note=vgl. S.123} um
% auf diese Keywords jetzt zuzugreifen wird im folgendem ein passender
% Befehl definiert. z.B. \citationAuthor der dann den Wert (hier
% Cosmina) erhält. Dies funktioniert durch den Aufruf von
% \pgfkeys{/citation/.cd, #1} in \citationtext. Weiter kann dann in
% \citationtext über die definierten Befehle \citationAuthor und
% \citationNote auf die übergebenen Werte also "Cosmina" und "vgl.
% S.123" zugreifen.
\pgfkeys{%
    /titledFramedEnvironments/.unknown/.code = { \relax },
    /titledFramedEnvironments/.cd,
        titlestyle/.code = {%
            \ifx#1\empty\else
                \def\titledFramedEnvironmentsTitlestyle{#1}
            \fi
        }
}

\newcommand{\setTitledFramePath}[1]{%
    \newcommand{\titledFramePath}{#1}
    \newcommand{\setFrameLabelStyle}{}
    \newcommand{\setFrameTitle}{} %definiert getFrameTitle
    \newcommand{\getFrameTitle}{} %in frame-title-styles definiert.
    \newcommand{\counterList}{}
    \newcommand{\setEnvironmentStyle}{}
}

\newcommand{\setEnvironment}[3][]{%
%
    % #1 ist ein Optionaler Parameter. Er enthält eine Liste von mit
    % einem Komma getrennten Key-Value Paaren, die mittels pgfkeys
    % passend benannte macros definieren, die im Verlaufe des Codes
    % dann an passenden Stellen ihre Anwendung finden.
    % Hier besonders relevant ist das Key-Value paar
    % titlestyle=default. Es beschreibt den Stil des Titels
    % (hauptsächlich wirkt sich das auf die
    % Nummerierung aus. Der Default wert des Paars ist default und
    % abgerufen kann der Wert mit dem Befehl \titledFramedEnvironmentsTitlestyle
    %
    %#2 ist der Name der Umgebung z.b. Definition oder Satz.
    %#3 ist der Stil der Umgebung, also z.B. light-blue oder
    %gray-white.
%
%
    % In folgendem wollen wir die optionale Variable weiter ausnutzen,
    % indem wir durch pgfkeys die Möglichkeit erhalten weitere Keyword
    % arguments anzugeben, wie zum Beispiel eine Referenz zu einem
    % Author oder ähnlichem. Hier interessiert uns aber nur der
    % Titelstil also wie das aussieht und nicht, der Inhalt, darum
    % wird sich in anderen Dateien wie z.B. Citationstyles gekümmert.
    \pgfkeys{/titledFramedEnvironments/.cd, #1}
    \ifdefined\titledFramedEnvironmentsTitlestyle\else
        \def\titledFramedEnvironmentsTitlestyle{default}
    \fi
    % Sollte \titledFramedEnvironmentsTitlestyle vor diesem Code nicht
    % definiert gewesen sein, weil er zum Beispiel in #1 nicht genannt
    % wurde, so ist er spätestens nach diesem Code mit dem Defaultwert
    % default definiert.
%
    \newcounter{#2}
    \listadd{\counterList}{#2}
    \input{./\titledFramePath/frame-title-label-styles/\titledFramedEnvironmentsTitlestyle}
    \setFrameLabelStyle{#2}
    \NewDocumentEnvironment{#2}{O{}m}
    %\newenvironment{#2}[2][] 
    %
        % ##1 enthält eine Liste von Key-Value Paaren wie zum Beispiel
        % titlestyle = default die sich mithilfe von pgfkeys auf
        % verschiedene Dinge auswirken. Der Wert von titlestyle definiert
        % den Befehl \titledFramedEnvironmentsTitlestyle neu. Dieser  
        % beschreibt den Stil des Titels (hauptsächlich wirkt sich das auf
        % die Nummerierung aus.) Der Befehl ist oben schon definiert,
        % sollte hier also nichts angegeben sein, so wird die
        % vorangegangene Definition verwendet. Die vorangegangene
        % Definition wird in der init Datei gesetzt.
        %
        %##2 ist der Inhalt des Titels.
        %
        % Für weitere informationen:
        % https://tex.stackexchange.com/questions/42463/what-is-the-meaning-of-double-pound-symbol-number-sign-hash-character-1-in
    %
    {
        %begin-code
        \pgfkeys{/titledFramedEnvironments/.cd, ##1}
        \input{./\titledFramePath/frame-title-styles/\titledFramedEnvironmentsTitlestyle}
        \setFrameTitle[##1]{#2}{##2}
        \input{./\titledFramePath/#3}
        \setEnvironmentStyle{#3}
        \begin{#3}
    }
    {
        %end-code
        \end{#3}
        \citationtext{##1}
    }
}
\newcommand{\resetTitledFrameEnvironmentCounters}{%
    \renewcommand{\do}[1]{%
        \setcounter{##1}{0}
    }
    \dolistloop{\counterList}
}

\newcommand{\setCodestylePath}[1]{%
    \newcommand{\codestylePath}{#1}
}

\newcommand{\setCodestyle}[1]{%
    \input{./\codestylePath/#1}
    \lstset{style=#1}
}

%\newcolumntype{L}{>{\raggedright \arraybackslash}X}
\newcolumntype{A}{>{\raggedright}p{0.85\linewidth}}
\newcolumntype{B}{>{\raggedleft}p{0.85\linewidth}}
\newcolumntype{R}{>{\raggedleft \arraybackslash}X}
\newcommand{\pA}[2]{
    \multicolumn{2}{A}{\textbf{#1}} & \\
    \multicolumn{2}{@{}A}{#2} & \\
}
\newcommand{\pB}[2]{
    & \multicolumn{2}{B}{\textbf{#1}}\\
    & \multicolumn{2}{B@{}}{#2} \\
}
\newenvironment{Diskussion}[1][]
{
    %begin-code
    \renewcommand{\arraystretch}{1.4}
    \xltabular[c]{\linewidth}{@{}Lp{0.7\linewidth}R@{}}
        %wenn Überschrift in #1: ansonsten nicht.
        \ifx &#1&
        \else
            \multicolumn{3}{@{}p{\linewidth}}{\textbf{#1}}\\
        \fi
        %\textbf{Pro} & & \textbf{Contra}\\
    \endfirsthead
    \endlastfoot
}
{
    %end-code
    \endxltabular
    \renewcommand{\arraystretch}{1}
}

\setlength{\columnsep}{1cm}
\newcommand{\multicollinenumbers}{%
    \linenumbers
    \def\makeLineNumber{\docolaction{\makeLineNumberLeft}{}{\makeLineNumberRight}}
}


\newcounter{tabs}
\setcounter{tabs}{0}
\newcounter{rows}
\setcounter{rows}{0}
\AtBeginEnvironment{xltabular}{
    \stepcounter{tabs}
    \setcounter{rows}{0}
}
% Besser nicht verwenden, Konflikt mit hypertarget und hyperlink
%\newcommand{\theTable}{
%    \arabic{tabs}
%}
%\newcommand{\theRow}{
%    \arabic{rows}
%}
\newcolumntype{\label}[1]{#1<{%
    \raisebox{1.5em}{\hypertarget{\arabic{tabs}:\arabic{rows}}{}}
    \phantomsection
    \stepcounter{rows}
}}
\newcommand{\reftab}[2]{%
    \hyperlink{#2}{#1}
}
\newcommand{\refctab}[2]{%
    \hyperlink{\arabic{tabs}:#2}{#1}
}

% Folgendes ist nur als Hilfestellung gedacht.
\newcolumntype{\numbered}[1]{>{%
    \arabic{rows} 
}#1}

\newcolumntype{\math}{
    @{\hspace{1pt}} >{$} c <{$} @{\hspace{1pt}}
    %@{}>{$} c <{$}
}

\newcolumntype{\pcenter}[1]{
    >{\centering} p{#1}
}


\setImagePath{lib/img}
\setPagestylePath{src/pagestyles}
\setLibraryPath{lib}
\setParaphrasingPath{src/paraphrasing}
\setCodestylePath{src/codestyles}
\setTitledFramePath{src/titled-frames}


\addlibresource
\setDefaultHeaderRuleWidth{1pt}
\setDefaultFooterRuleWidth{1pt}

\AfterEndPreamble{
    \begin{titlepage}
    \begin{center}
        \vspace*{1cm}
        \Huge
        \textbf{Exakte Differenzialgleichungen}\\
        \LARGE
        \vspace{0.5cm}
        Eine Ausarbeitung zu den Beispielen 4.1.3, 
        4.1.4 und 4.1.5 \source{Aulbach}{S.140, 142 und 144} 

        \vspace{1.5cm}
        \textbf{Jana Briggl \vspace{-1.1cm} \\ \hspace{2.75cm} \footnotesize \footnotemark}
        \footnotetext{%
            \vspace{-0.6cm}
            \setlength\LTleft{1cm}
            \begin{longtable}{ll}
                Matrikelnummer  &   12004882\\
                E-Mail-Adresse    &
                \href{mailto:jabriggl@edu.aau.at}{jabriggl@edu.aau.at}
            \end{longtable}
        }\newline
        \textbf{Markus Teichmann \vspace{-1.1cm} \\ \hspace{4.5cm} \footnotesize \footnotemark}
        \footnotetext{%
            \vspace{-0.8cm}
            \setlength\LTleft{1cm}
            \begin{longtable}{ll}
                Matrikelnummer  &   12024261\\
                E-Mail-Adresse    &   \href{mailto:markuste@edu.aau.at}{markuste@edu.aau.at}
            \end{longtable}
        }

        \Large
        %\vspace{0.5cm}

        %\vfill
        \vspace{3cm}

        %\vspace{0.8cm}
        \begin{figure}[htp]
            \centering
            \includesvg{7cm}{Universitaet_klagenfurt_logo}
        \end{figure}
        Fakultät für Technische Wissenschaften\\
        Institut für Mathematik\\
        Alpen Adria Universität Klagenfurt\\
        Österreich\\
        \datumVonHeute

    \end{center}
\end{titlepage}
 \label{in:titlepage}      % Natürlich nur, wenn man eine Titlepage.
    \setThisPagestyle{toc}                      % Natürlich nur, wenn toc etwünscht.
    \tableofcontents{}                            % Natürlich nur, wenn toc etwünscht.
    \newpage                                    % Notwendig.

    %Letzte Einstellungen, welche dann für das gesamte Dokument gelten sollen.
    %\setEnvironment{Sagemath}{gray-white}     % Alle möglichen Kästen die wir mit den Stilen bauen können.
    %\setCodestyle{default}                     % Stil des Quellcodes, falls wir welchen verwenden.
    \setParaphrasing{AAU}          % Hier geht es um die Überschriften etc.
    %\setThisPagestyle{Aufgabenblatt}           % Header und Footer - Lokal
    \setPagestyle{AAU}                      % Header und Footer - global
    %\captionsetup[figure]{labelformat=empty}   % Falls nötig, so lässt sich damit die Beschriftung von Abbildungen umgehen
    \setEnvironment[default]{Beispiel}{light-lila}  % Neue Umgebung.
    \setEnvironment[default]{Aufgabe}{light-gray}

    \usetikzlibrary{calc}
    \usetikzlibrary{intersections}
    \usetikzlibrary{angles}
    \usetikzlibrary{quotes}
}

\AtEndDocument{
    \lit
    %\anhang
}
