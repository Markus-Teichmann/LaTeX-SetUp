\documentclass[12pt]{scrartcl}%{article} % Beginn der LaTeX-Datei

%% twocolumn

\usepackage{amsmath,amssymb}  % erleichtert Mathe 
\usepackage{enumerate}% schicke Nummerierung

\usepackage{graphicx} % für Grafik-Einbindung
%\usepackage{hyperref}

\usepackage[ngerman]{babel}
\usepackage[T1]{fontenc}
\usepackage{lmodern}
 % Einstellungen, wenn man deutsch schreiben will, z.B. Trennregeln
\usepackage[utf8]{inputenc}  % für Unix-Systeme
  % ermöglicht die direkte Eingabe von Umlauten und ß
  % evt. obige Zeile ersetzen durch
  % \usepackage[ansinew]{inputenc}  % für Windows
  % \usepackage[applemac]{inputenc} % für den Mac


%%%%%%%%%%%%%%%%%%%%%%%%%%%%%%%%%%%%%%%%%%%%%%%%%%%%%%%%%%%%%%%%%%
%
%  ntheorem
%
\usepackage[thmmarks,amsmath,hyperref,noconfig]{ntheorem} 
  % erlaubt es, Sätze, Definitionen etc. einfach durchzunummerieren.
\newtheorem{satz}{Satz}[section] % Nummerierung nach Abschnitten
\newtheorem{hilfssatz}[satz]{Hilfssatz}
\newtheorem{kor}[satz]{Korollar}

\theorembodyfont{\upshape}
\newtheorem{beispiel}[satz]{Beispiel}
\newtheorem{bemerkung}[satz]{Bemerkung}
\newtheorem{definition}[satz]{Definition} %[section]

\theoremstyle{nonumberplain}
\theoremheaderfont{\itshape}
\theorembodyfont{\normalfont}
\theoremseparator{.}
\theoremsymbol{\ensuremath{_\blacksquare}}
\newtheorem{beweis}{Beweis}
\qedsymbol{\ensuremath{_\blacksquare}}
%\theoremclass{LaTeX}
%
% Ende ntheorem
%
%%%%%%%%%%%%%%%%%%%%%%%%%%%%%%%%%%%%%%%%%%%%%%%%%%%%%%%%%%%%%%%%%%


%\pagestyle{empty}
%
% Ändern der bedruckten Fläche der Seite
% \addtolength{\textwidth}{3cm}  % Befehl mit zwei Argumenten
% \addtolength{\textheight}{3cm}
% \hoffset-2cm % verschiebt das Textfenster nach links
% \voffset-5mm % verschiebt das Textfenster nach oben
%
%\parindent=0pt %% keine Einzug zu Beginn von Abs\"atzen
%\parskip=2mm   %% erzeugt einen zusätzliche Zeilenabstand zwischen
                %% Absätzen. Nötig bei \parindent=0pt


%%%%%%%%%%%%%%%%%%%%%%%%%%%%%%%%%%%%%%%%%%%%%%%%%%%%%%%%%%%%%%%%%%
% ermöglicht, farbigen Text zu drucken.
\usepackage{color}
% Und nun werden die Farben definiert - daran können Sie nach Belieben selber rumspielen.
\definecolor{white}{rgb}{1,1,1}
\definecolor{darkred}{rgb}{0.3,0,0}
\definecolor{darkgreen}{rgb}{0,0.3,0}
\definecolor{darkblue}{rgb}{0,0,0.3}
\definecolor{pink}{rgb}{0.78,0.09,0.51}
\definecolor{purple}{rgb}{0.28,0.24,0.55}
\definecolor{orange}{rgb}{1,0.6,0.0}
\definecolor{grey}{rgb}{0.4,0.4,0.4}
\definecolor{aquamarine}{rgb}{0.4,0.8,0.65}


\DeclareMathOperator{\GL}{GL} % einige Macro, siehe am Ende Abschn. 2
\newcommand{\N}{\mathbb{N}}
\newcommand{\Z}{\mathbb{Z}}
\newcommand{\Q}{\mathbb{Q}}
\newcommand{\R}{\mathbb{R}}
\newcommand{\C}{\mathbb{C}}
\newcommand{\cP}{{\mathcal P}} 

\begin{document}

\author{Hubert Kiechle}
\title{Beispiel einer LaTeX-Datei}
\date{} %hier können Sie ein Datum eingeben, auch leer, sonst wird es
         %automatisch erzeugt

\maketitle % erzeugt den Kopf


\section*{Einleitung}  % *= ohne Nummer

In dieser Datei sind einige Beispiele für Anwendungen von LaTeX
enthalten. Es handelt sich aber keineswegs um eine echte Einführung.
Einziges Ziel ist es eine Vorlage zu liefern, die als Gerüst für eine
in LaTeX verfasste Ausarbeitung dienen kann. Dazu finden sich Beispiele
für Formatierungsbefehle und den Mathematiksatz.

Wichtig ist die Tatsache, dass der eingetippte Text
\glqq übersetzt\grqq\ werden muss. Um diese Datei zu nutzen ist es
zweckmäßig den Quelltext mit der Übersetzung zu vergleichen, um einige
Effekte zu verstehen.

Das Zeichen \% % jfgakerjrgnfaLV
im Quelltext markiert einen Kommentar. Alles was hinter diesem Zeichen
in derselben Zeile steht, wird beim \"ubersetzen ignoriert.

Befehle beginnen immer mit $\backslash$, etwa $\backslash$hspace,
Argumente werden mit $\{$ und $\}$ geklammert.


Jede LaTeX-Datei beginnt mit einer Prämbel in der nur Befehle
eingegeben werden können. Die Bedeutung der benutzten Befehle in
dieser Datei ist in der LaTeX-Datei als Kommentar kurz angedeutet.

\begin{itemize}
\item $\backslash$begin$\{$document$\}$ markiert den Punkt ab dem Text
eingegeben werden kann.
\item $\backslash$end$\{$document$\}$ beendet den Text. Was nach diesem
Befehl steht, wird beim übersetzen ignoriert.
\end{itemize}


Umlaute erhält man durch \"a, wenn (wie hier) das Packet ngerman
geladen ist durch "a, und wenn  (wie hier) das Packet ``inputenc'' mit
``latin1'' (oder ``utf8'')geladen ist auch durch ä. Entsprechend \ss, "s, ß


\section{Text ...}

... wird einfach eingetippt. Leerzeichen trennen wie üblich Wörter,
{\color{blue}die Anzahl ist        egal.}
Auch Zeilenumbrüche im Quelltext werden ignoriert.
Zeilenumbrüche macht das System
selbständig. Ein neuer Absatz entsteht, wenn man im Quelltext eine

Zeile frei lässt.  

Um Wortabstände und Zeilenabstände künstlich zu verändern gibt es die
Befehle
$\backslash$hspace \hspace{9mm} (siehe Quelltext) und 
$\backslash$vspace \vspace{10mm}
die man aber selten braucht. Man beachte, dass der zweite Befehl erst
beim folgenden Zeilenumbruch wirkt. Er wird eigentlich nur zwischen

\vspace{10mm}
Absätzen und im Mathematiksatz eingesetzt.\\  % neue Zeile ohne Absatz
\textbf{Empfehlung:} Sparsam (oder gar nicht) benutzen!

\subsection{Schriftgröße}

Man kann die Größe der Buchstaben ändern durch:

{\large groß}, {\LARGE größer}, {\huge noch größer} und {\small klein}
oder {\tiny winzig} 

\subsection{Schriftart}

Um im Text Teil hervorzuheben kann man 

\textit{kursive Schrift}
\quad % = Abstand ca. 3x Leerzeichen
\textbf{fett}
\quad
\texttt{Schreibmaschinen-Schrift} 
\quad
\textsc{Kapitälchen}
\quad
\textbf{ \textit{Test}}

benutzen.

\paragraph{Farben}
\textcolor{darkred}{gehen} \textcolor{darkgreen}{so}.
Dazu braucht man aber die Definitionen aus der Präambel.

\subsection*{Strukturierung}

Möglichkeiten zur Strukturierung  erkennen Sie im obigen Text. Dazu
müssen Sie stets den Quelltext mit dem Ergebnis des Übersetzens
vergleichen.

Die wichtigesten Befehle sind

\bigskip % vergrößerter Zeilenabstand, auch \smallskip, \medskip

\begin{tabular}{lr|c} %l=left; auch c=center, r=right
Befehl & Nummerierung & Bemerkung \\
\hline
  $\backslash$section$\{$Überschrift$\}$ 
& ja
\\  
$\backslash$section*$\{$Überschrift$\}$  
& nein
\\ $\backslash$subsection$\{$Unterüberschrift$\}$ & ja
\\ $\backslash$subsection*$\{$Unterüberschrift$\}$
& nein
\\ $\backslash$paragraph$\{$Überschrift$\}$  
& nein & keine neue Zeile
\end{tabular}

\paragraph{Kommentare zur Tabelle:}
$\{$lr|c$\}$ bedeutet, dass die erste Spalte linksbündig, die zweite
Spalte rechtsbündig, dann eine Linie und die dritte Spalte zentriert
gesetzt wird.

Das Zeichen \& wird als Tabulator benutzt, $\backslash\backslash$
markiert das Ende der Zeile.


\section{Mathematik}

Mathematische Symbole und Formeln werden immer zwischen \$ gesetzt;
etwa $P$ oder $(P,\mathfrak{G})$ und nicht P. Will man eine Formel
absetzen, so
schreibt man 
\[
\sin^{1+1}x+\cos^2x=1 \mbox{\quad für alle } x\in\mathbb R\ \ nicht so
\]
Wichtig sind \glqq Umgebungen\grqq. Einige Beispiele
\begin{equation}
  \label{test} % setzt eine Marke
  \int_1^\infty\frac{1}{x^2}=\arctan\frac\pi4
=\sum_{k=1}^\infty\left(\frac{1}{2}\right)^k
\end{equation}
liefert eine numerierte Gleichung,  auf die durch 
\eqref{test} auch \ref{test} Bezug 
genommen werden kann. Beachte die Benutzung von
$\backslash$left und $\backslash$right 
um die Klammern automatisch(!) auf die richtige Größe zu bringen.
Gleichungssysteme erhält man durch
  \begin{eqnarray*} % *= ohne Nummerierung
    \langle m,c\rangle&:=&\{(x_1,x_2)\in K^2;\; x_2=mx_1+c\} \\
    \langle c\rangle&:=&\{(x_1,x_2)\in K^2;\; x_1=c\}
  \end{eqnarray*}
ohne Nummern, oder mit
\begin{eqnarray}
  \label{eq:test2} % Marke für erste Gleichung
  S^2 &:= &\left\{(x_1,x_2,x_3)\in\mathbb R^3;\,
x_1^2+x_2^2+(x_3-\frac12)^2=\frac14\right\}\\
  \label{eq:test3} % Marke für zweite Gleichung
S^-&:=&\left\{(x_1,x_2,x_3)\in S^2;\,x_3<\frac12\right\},
\end{eqnarray}
Das Zeichen \& wird als Tabulator benutzt, $\backslash\backslash$
markiert das Ende der Zeile.
\\ 
Nach Gl.~\eqref{eq:test3} gilt \dots
\qquad
--- vgl. mit \ref{eq:test3}

Um mathematische Standard-Abkürzungen richtig zu setzen schreibt man

$\mbox{GL}(n,K)$ nicht etwa $GL(n,K)$.

Noch besser ist es ein Macro zu definieren, d.\,i.\ ein
selbstdefinierter Befehl:

% \DeclareMathOperator{\GL}{GL} 
% der ist in der Präambel zu finden
% weil er hier nicht stehen kann!

Dann kann man $\GL(n,K)$ schreiben.

Das ist dann schon etwas für Experimentierfreudige oder für
Fortgeschrittene.
% (aber wie sollte man ohne Experimentierfreude vorwärts kommen?)


\paragraph{Vordefiniert sind} 
$\N, \Z, \Q, \R, \C, \cP$

Weitere können Sie leicht selbst definieren.





\section{Zeichen und Listen}

Einige Beispiele

\begin{itemize}
\item \textit{Griechische Buchstaben:} $\alpha, \beta, \gamma, \dots, \phi,
\psi$ und große $\Gamma, \Delta, \dots, \Phi, \Psi$

\item \textbf{Einige Zeichen:} $\cap, \cup, \subseteq, \setminus,
  \times, \otimes, \le, \ge, \in, \parallel, \perp, \angle$
\end{itemize}
und eine nummerierte Liste
\begin{enumerate}
\item in der 
\item aber
\item nichts steht
  \begin{enumerate}
  \item Dennoch hat sie 
  \item eine Unterliste
  \end{enumerate}
\item 
  \begin{itemize}
  \item Die nicht nummeriert sein muss
    \begin{itemize}
    \item und ihrerseite Unterlisten haben kann
    \item usw.
    \end{itemize}
  \end{itemize}

\end{enumerate}

Wenn man (wie hier) das Packet ``enumerate'' geladen hat, kann man die
Ausgestaltung der Nummerierung leichter gestalten.

\begin{enumerate}[\bfseries (E1)]
\item M\"oglich sind
  \begin{enumerate}[A.)]
  \item 1
  \item a
  \item A
  \item i
  \item I
  \end{enumerate}
  \item jeweils mit Klammern, Punkten, Kommata usw.
\end{enumerate}

\begin{description}
\item [Test] hier kommt der Text
\item[Test II] nächster Punkt
\item[Item] 
\end{description}

\section{Satz --- Beweis}

Mathematische Texte sind durch Definition, Satz, Beweis usw. tiefer
strukturiert als typische andere Texte. Und so kann man das machen:

\begin{definition}
  Dies ist eine Definition.
\end{definition}


\begin{satz}\label{s:satz1}
  So sieht dann ein Satz aus.
\end{satz}

Und der kann referenziert werden durch Satz~\ref{s:satz1} 

\begin{beweis}
  Hier steht der Beweis.
\end{beweis}

Auch

\begin{beweis}[von Satz~\ref{s:satz1}]
  wenn man später darauf zurück kommen will.
\end{beweis}


\begin{bemerkung}
  %Wir bemerken nichts weiter!
So kann man Literatur zitieren, nach \cite{Heu03} gilt. Oder auch
nach  \cite[35.6]{Heu03}. 

\cite{GM-GM}
\end{bemerkung}


\section{Satznummerierung}

Im neuen Abschnitt wird neu nummeriert; oder fortlaufende, wenn
\texttt{[section]} nicht gesetzt ist.

\begin{satz}[von W. Ichtig]
  Noch eine Satz.
\end{satz}


\section{Grafiken}

Damit die Grafik läuft müssen Sie die Datei  \texttt{graph.pdf}
herunterladen und in das Verzeichnis mit \texttt{Beispiel.tex}
kopieren. 

 % \begin{center}
 %   \includegraphics[scale=.8]{graph.pdf}
 % \end{center}

% um es auszuprobieren: Kommentarzeichen entfernen!
% geht so nur mit pdf-LaTeX
% man kann auch  jpg Graphiken benutzen

\begin{thebibliography}{99}
\bibitem{Heu03}
 \textsc{Heuser}, Harro:
  \newblock \emph{Lehrbuch der Analysis}.
  \newblock 15. Aufl.
  \newblock Vieweg-Verlag, Braunschweig-Wiesbaden, 2003

\bibitem{GM-GM}
\textsc{Gr{\"o}ger}, Detlef ; \textsc{Marti}, Kurt:
\newblock \emph{Grundkurs Mathematik für Ingenieure, Natur- und
  Wirtschaftswissenschaftler}.
\newblock 2.~Aufl.
\newblock Physica-Verlag, 2004
\end{thebibliography}

\end{document}
